\documentclass{jhwhw}
\usepackage{amssymb,amsmath,graphicx}

\begin{document}

\title{Computer Vision - Assignment Two}
\author{Cody Boppert}

\maketitle

\section*{Assignment Two}
\nopagebreak[4]

\problem{Question 1}
1. (20 points) Generate the histogram of the image you are using .  If it is a color image,  please first turn it into an intensity image and then generate its histogram.  Try to display your histogram, and make some observations of the image based on its histogram.

\solution
\part
Original Image:
{\centering
	\includegraphics[width=500px]{../../Images/AssignmentTwo/01_original.png}\par
}
\part
Intensity Image
{\centering
	\includegraphics[width=500px]{../../Images/AssignmentTwo/02_intensity.png}\par
}

\part
Histogram of Intensity Image:
{\centering
	\includegraphics[width=500px]{../../Images/AssignmentTwo/03_histogram.png}\par
}

\problem{Question 2}
Apply the 1x2 operator and Sobel operator to your image and analyze the results of the gradient magnitude images.  Does the Sobel operator have any clear visual advantages over the 1x2 operator?  If you subtract the 1x2 edge image from the Sobel are there any residuals?

\solution
\part
2x1 Operator:
{\centering
	\includegraphics[width=500px]{../../Images/AssignmentTwo/04_2x1.png}\par
}
\part
1x2 Operator:
{\centering
	\includegraphics[width=500px]{../../Images/AssignmentTwo/05_1x2.png}\par
}
\part
Combined 1x2 and 2x1 Operators (Gradient):
{\centering
	\includegraphics[width=500px]{../../Images/AssignmentTwo/06_combinedGradientOperators.png}\par
}
\part
Sobel 3x3:
{\centering
	\includegraphics[width=500px]{../../Images/AssignmentTwo/07_sobel3x3.png}\par
}
\part
Sobel Minus Gradient Operator:
{\centering
	\includegraphics[width=500px]{../../Images/AssignmentTwo/08_sobel_minus_gradient.png}\par
}
\part
Sobel Histogram:
{\centering
	\includegraphics[width=500px]{../../Images/AssignmentTwo/09_sobelHistogram.png}\par
}
\part
Gradient Histogram:
{\centering
	\includegraphics[width=500px]{../../Images/AssignmentTwo/10_gradientHistogram.png}\par
}
\part
Sobel Minus Gradient Histogram:
{\centering
	\includegraphics[width=500px]{../../Images/AssignmentTwo/11_sobelMinusGradientHistogram.png}\par
}

\problem{Question 3}
Generate edge maps of the above gradient maps (20 points).  You may first generate a histogram of each gradient map,  and only keep certain percentage of pixels  (e.g.  5\% of the highest gradient  values) as edge pixels (edgels) . Use the percentage to find a threshold for the gradient magnitudes. 

\solution
\part
Sobel Edge Map 3 Percent of Values:
{\centering
	\includegraphics[width=500px]{../../Images/AssignmentTwo/12_sobelEdgeMap3percent.png}\par
}
\part
Sobel Edge Map 5 Percent of Values
{\centering
	\includegraphics[width=500px]{../../Images/AssignmentTwo/14_sobelEdgeMap5percent.png}\par
}
\part
Sobel Edge Map 7 Percent of Values
{\centering
	\includegraphics[width=500px]{../../Images/AssignmentTwo/16_sobelEdgeMap7percent.png}\par
}
\part
Sobel Edge Map 9 Percent of Values
{\centering
	\includegraphics[width=500px]{../../Images/AssignmentTwo/18_sobelEdgeMap9percent.png}\par
}
\part
Gradient Edge Map 3 Percent of Values:
{\centering
	\includegraphics[width=500px]{../../Images/AssignmentTwo/19_gradientEdgeMap3percent.png}\par
}
\part
Gradient Edge Map 5 Percent of Values:
{\centering
	\includegraphics[width=500px]{../../Images/AssignmentTwo/21_gradientEdgeMap5percent.png}\par
}
\part
Gradient Edge Map 7 Percent of Values
{\centering
	\includegraphics[width=500px]{../../Images/AssignmentTwo/23_gradientEdgeMap7percent.png}\par
}
\part
Gradient Edge Map 9 Percent of Values
{\centering
	\includegraphics[width=500px]{../../Images/AssignmentTwo/25_gradientEdgeMap9percent.png}\par
}
\part
Sobel Minus Gradient Edge Map 3 Percent of Values:
{\centering
	\includegraphics[width=500px]{../../Images/AssignmentTwo/26_sobelMinusGradientEdgeMap3percent.png}\par
}
\part
Sobel Minus Gradient Edge Map 5 Percent of Values:
{\centering
	\includegraphics[width=500px]{../../Images/AssignmentTwo/28_sobelMinusGradientEdgeMap5percent.png}\par
}
\part
Sobel Minus Gradient Edge Map 7 Percent of Values:
{\centering
	\includegraphics[width=500px]{../../Images/AssignmentTwo/30_sobelMinusGradientEdgeMap7percent.png}\par
}
\part
Sobel Minus Gradient Edge Map 9 Percent of Values:
{\centering
	\includegraphics[width=500px]{../../Images/AssignmentTwo/32_sobelMinusGradientEdgeMap9percent.png}\par
}

\problem{Question 4}
What happens when you increase the size of the kernel to 5x5 , or 7x7? Discuss computational cost (in terms of members of operations, and the real machine running times), edge detection results and sensitivity to noise, etc. Note that your larger kernel should still be an edge detector.

\solution
\part
Sobel 5x5:
{\centering
	\includegraphics[width=500px]{../../Images/AssignmentTwo/33_sobel5x5.png}\par
}
\part:
Sobel 7x7:
{\centering
	\includegraphics[width=500px]{../../Images/AssignmentTwo/34_sobel7x7.png}\par
}
\part
Sobel 9x9:
{\centering
	\includegraphics[width=500px]{../../Images/AssignmentTwo/35_sobel9x9.png}\par
}

\problem{Question 5}
Suppose you apply the Sobel operator to each of the RGB color planes comprising the image.  How might you combine these results into a color edge detector?  Do the resulting edge differ from the gray scale results?  How and why?

\solution
\part
Sobel Difference Red:
{\centering
	\includegraphics[width=500px]{../../Images/AssignmentTwo/36_sobelDifferenceRed.png}\par
}
\part
Sobel Difference Green:
{\centering
	\includegraphics[width=500px]{../../Images/AssignmentTwo/37_sobelDifferenceGreen.png}\par
}
\part
Sobel Difference Blue:
{\centering
	\includegraphics[width=500px]{../../Images/AssignmentTwo/38_sobelDifferenceBlue.png}\par
}
\part
Sobel Difference RGB:
{\centering
	\includegraphics[width=500px]{../../Images/AssignmentTwo/39_sobelDifferenceRGB.png}\par
}
\part
Sobel Difference RGB to Gray:
{\centering
	\includegraphics[width=500px]{../../Images/AssignmentTwo/40_sobelDifferenceRGBGray.png}\par
}
\end{document}